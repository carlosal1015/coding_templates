\documentclass[12pt, a4paper, twopage]{scrartcl}

%\documentclass[
%11pt, % The default document font size, options: 10pt, 11pt, 12pt
%oneside, % Two side (alternating margins) for binding by default, uncomment to switch to one side
%chapterinoneline,% Have the chapter title next to the number in one single line
%english, % ngerman for German
%singlespacing, % Single line spacing, alternatives: onehalfspacing or doublespacing
%draft, % Uncomment to enable draft mode (no pictures, no links, overfull hboxes indicated)
%nolistspacing, % If the document is onehalfspacing or doublespacing, uncomment this to set spacing in lists to single
%liststotoc, % Uncomment to add the list of figures/tables/etc to the table of contents
%toctotoc, % Uncomment to add the main table of contents to the table of contents
%parskip, % Uncomment to add space between paragraphs
%nohyperref, % Uncomment to not load the hyperref package
%headsepline, % Uncomment to get a line under the header
%]{scrartcl or scrreprt or scrbook} % The class file specifying the document structure

%-----------------------------------------------
% LOADING PACKAGES
%-----------------------------------------------

\usepackage{lmodern} 		% Diese beiden packages sorgen für echte 
\usepackage[T1]{fontenc}	% Umlaute.

\usepackage{amssymb, amsmath, color, graphicx, float, setspace, tipa}
\usepackage[utf8]{inputenc} 
\usepackage[ngerman]{babel}
\usepackage[pdfpagelabels,pdfstartview = FitH,bookmarksopen = true,bookmarksnumbered = true,linkcolor = black,plainpages = false,hypertexnames = false,citecolor = black, breaklinks]{hyperref}
\usepackage{url}
\usepackage{picins} 		%Gleittext um Grafik. Befehl: parpic. Vorlage siehe unten
\usepackage{longtable} 		%Seitenübergreifende Tabelle. Vorlage siehe unten
\usepackage{caption}
\captionsetup{font=small,labelfont=bf, format=plain, justification=centering}
\allowdisplaybreaks % allows page breaks in align/equation environment

%bibliography
\usepackage[backend=bibtex,sorting=nyt,bibencoding=ascii]{biblatex}
\addbibresource{references.bib}
\usepackage{csquotes} %recommended when using babel and biblatex

%glossary
%\usepackage[toc,acronyms]{glossaries} % glossary. Set after usepackage{hyperref}
\usepackage{glossaries} % glossary. Set after usepackage{hyperref}
\makeglossaries

% IMPORTANT NOTE:
% Don't forget to recompile the glossary and the bibliography first if you want them to show up!

%-----------------------------------------------
% TEXT IN BOXEN
%-----------------------------------------------

\usepackage[framemethod=TikZ]{mdframed}
\mdfdefinestyle{exampledefault}{%
	rightline=true,innerleftmargin=10,innerrightmargin=10,
	frametitlerule=true,frametitlerulecolor=green,
	frametitlebackgroundcolor=yellow,
	frametitlerulewidth=2pt}
\mdfdefinestyle{simple}{%
	%rightline=true,
	innerleftmargin=10,
	innerrightmargin=10,
	%frametitlerule=true,
	%frametitlerulecolor=black,
	frametitlebackgroundcolor=lightgray,
	frametitlerulewidth=2
	}

%--------------------------------------------
% NEUE BEFEHLE
%--------------------------------------------
% Gleich mit Dach obendrauf
\newcommand{\entspricht}{\mathrel{\widehat{=}}}

% Atanh
\newcommand {\arctanh}{\mathrm{arctanh}}

% Acotanh
\newcommand{\arccot}{\mathrm{arccot }}

% Limes von etwas gegen null
\newcommand{\limz}[1]{\lim\limits_{#1 \rightarrow 0}}

%Bold font in math
\newcommand{\bm}{\boldmath}
\newcommand{\dps}{\displaystyle}

% e noncursive in math mode
\newcommand{\e}{\mbox{e}}

% partial diff operator
\newcommand{\del}{\partial}


%-----------------------------------------------
% TITEL FORMATIEREN
%-----------------------------------------------


% Set fonts of document parts
\setkomafont{title}{\rmfamily\bfseries\boldmath}
\addtokomafont{section}{\rmfamily\bfseries\boldmath}
\addtokomafont{subsection}{\rmfamily\bfseries\boldmath}
\addtokomafont{subsubsection}{\rmfamily\bfseries\boldmath}
\addtokomafont{disposition}{\rmfamily} % table of contents and stuff
\setkomafont{descriptionlabel}{\rmfamily\bfseries\boldmath}


%--------------------------------------------
%--------------------------------------------
%---------OPTIONAL---------------------------

%% Schriftart ändern
%\newcommand{\changefont}[3]{
%\fontfamily{#1} \fontseries{#2} \fontshape{#3} \selectfont}
%\changefont{ppl}{m}{n} nach \begin{document} einsetzen

%% Abb. statt Abbildung, Tab. statt Tabelle
%\usepackage[footnotesize]{caption2}
\addto\captionsngerman{\renewcommand{\figurename}{Abb.}}
%\renewcommand{\tablename}{Tab.}%

%\pagestyle{headings} % Überschrift an jeder Seite

%\usepackage{chngcntr} \counterwithout{figure}{section} % Ganzzahlige Bildnummerierungen, Kapitelunabhängig



%-------------
% Set paper margins
%-------------

%\usepackage{geometry}
%\geometry{
%	a4paper,
%	total={170mm,257mm},
%	left=20mm,
%	top=20mm,
%}




% Set fonts of document parts
%\setkomafont{title}{\rmfamily\bfseries\boldmath}
%\addtokomafont{section}{\rmfamily\bfseries\boldmath}
%\addtokomafont{subsection}{\rmfamily\bfseries\boldmath}
%\addtokomafont{subsubsection}{\rmfamily\bfseries\boldmath}
%\addtokomafont{disposition}{\rmfamily} % table of contents and stuff
%\setkomafont{descriptionlabel}{\rmfamily\bfseries\boldmath}









%------------------------------------------
%:Metainformationen

\title{Titel}
\author{Mladen Ivkovic\\
mladen.ivkovic@uzh.ch\\
}
\date{Datum}

%------------------------------------------











\begin{document}
		
		
%--------------------------------------------
% Stuff that needs to be done before all else
%--------------------------------------------
%\pagestyle{plain}
\nocite{*} % show all entries of bibliography, even if they are not cited.
		
		
%\pagestyle{plain}





%Titlepage

\maketitle
\clearpage

%Table of Contents
\tableofcontents %Auf englisch wechseln: Ändere usepackage ngeman babel in english babel
\clearpage







\section*{Anmerkung des Autoren}

Dieser Abschnitt ist nicht nummeriert und nicht im Inhaltsverzeichnis. 


\begin{description}
  \item[Zweck] Dieses Dokument blablabla.
  \item[Punkt 2] Punkt 2
\end{description}

Sonstiger text: Bla blablabla blabla bla. Blabla bla. Blablablabal basdiga asdifsdjfh asdfjlsdfn uilsdfyjkzu shflsdf jhksdfui sf df,jhi afuil sdfuinm,j shsdfnm,,.
\clearpage






















\section{Kapitel 1}
\subsection{Unterkapitel 1.1}
\subsubsection{Unterunterkapitel 1.1.1}

%
\piccaption{Darstellung des Zahlenbereichs des Zweierkomplements mit acht Stellen\label{fig:tabelle_zweierkomplement}}
\parpic[r]{%
  \fbox{
    \includegraphics[width = 3.5cm, keepaspectratio]{images/random_image.png}
  }
}
%
Die gängigste Form der Zahlensysteme sind Stellenwertsysteme. Eine Zahl $a$ wird in Form einer Reihe von Ziffern $z_i$ mit dazugehöriger Potenz der Basis $b^i$ dargestellt. Der Wert der Zahl ergibt sich dann als Summe der Werte aller Einzelstellen: $a = \sum\limits_{i}z_ib^i$.

\textbf{Umrechnung} in andere Zahlensysteme: Gegeben sei Zahl $Z$, umzuwandeln in System mit Basis $b$.
Eine angenehme Vorgehensweise gibt uns das \textbf{Horner Schema}\footnote{
Website mit Umrechnungen und Erklärungen: \url{http://www.arndt-bruenner.de/mathe/scripts/Zahlensysteme.htm}
}: Dividiere $Z$ durch $b$. Der Rest dieser Division ist die letzte Stelle der Zahl in der Basis $b$  (Einerstelle). Dividiere den Quotienten dieser Division wieder durch $b$. Der Rest dieser zweiten Division ergibt die zweite Stelle der Zahl in der neuen Basis. Wiederhole Divisionen, bis kein Rest mehr.


\section{Tabellen}
\subsection{Einfach}
\begin{center}
\begin{tabular}[c]{c | c | c || c| c | c || c | c || c | c | c || c| c| c}
\multicolumn{3}{c||}{Konjunktion}	&	\multicolumn{3}{c||}{Disjunktion} & \multicolumn{2}{c||}{Negation} & \multicolumn{3}{c||}{NAND} & \multicolumn{3}{c}{NOR}\\
\multicolumn{3}{c||}{UND}	&	\multicolumn{3}{c||}{ODER} & \multicolumn{2}{c||}{} & \multicolumn{3}{c||}{} & \multicolumn{3}{c}{}\\
\hline
$a$ & $b$ & $a$ $\wedge$ $b$ & $a$ & $b$ & $a$ $\vee$ $b$ & $a$ & $\bar{a}$ & $a$ & $b$ & $\overline{a \wedge b}$ & $a$ & $b$ & $\overline{a \vee b}$\\
\hline
0 & 0 & 0 & 0 & 0 & 0 & 0 & 1 & 0 & 0 & 1 & 0 & 0 & 1\\
0 & 1 & 0 & 0 & 1 & 1 & 1 & 0 & 0 & 1 & 1 & 0 & 1 & 0\\
1 & 0 & 0 & 1 & 0 & 1 & & & 1 & 0 & 1 & 1 & 0 & 0\\
1 & 1 & 1 & 1 & 1 & 1 & & & 1 & 1 & 0 & 1 & 1 & 0\\
\hline
\end{tabular}
\end{center}

\subsection{Mit extra + Seitenumbruch}
\begin{spacing}{1.4}
\begin{longtable}{p{4cm} l l}
Kommutativgesetz: & $a \wedge b = b \wedge a$ & $a \vee b = b \vee a$\\ 
Distributivgesetz: & $a \wedge (b \vee c) = (a \wedge b) \vee (a \wedge b)$ & $a \vee (b \wedge c) = (a \vee b) \wedge (a \vee c)$\\ 
Neutrales Element & $a \wedge 1 = a$ & $a \vee 0 = a$\\
Inverses Element & $a \wedge \bar{a} = 0$ & $a \vee \bar{a} = 1$\\
Assoziativgesetz & $(a \wedge b) \wedge c = a \wedge (b \wedge c)$ & $(a \vee b) \vee c = a \vee (b \vee c)$\\
Idempotenzgesetz & $a \wedge a = a$ & $a \vee a = a$\\
Absorptionsgesetz & $a \wedge ( a \vee b) = a$ & $a \vee (a \wedge b) = a$\\
DeMorgan-Gesetz & $\overline{a\wedge b} = \bar{a} \vee \bar{b}$ (NAND)& $\overline{a \vee b} = \bar{a} \wedge \bar{b}$ (NOR)\\
\smash{Gesetz vom Widerspruch} & & $a \wedge \overline{a} = 0$\\
\smash{Gesetz vom ausgeschl. Dritten} & & $a \vee \overline{a} = 1$ \\
\smash{Gesetz der doppelten Negation} & & $\overline{\overline{a}} = a$ \\
\end{longtable}
\end{spacing}














%-----------------------------
\section{Zwei Bilder}

\begin{figure}[h!]
\centering
  \minipage{0.3\textwidth}
  \centering
    \fbox{\includegraphics[height=2.5cm, keepaspectratio]{images/random_image.png}}%
    \caption{RS-Flipflop}%
    \label{fig:rsflipflop}
  \endminipage\hspace{1cm}   
%
  \minipage{0.4\textwidth}
	\centering
    \fbox{\includegraphics[height=2.5cm, keepaspectratio]{images/random_image_2.png}}%
    \caption{getaktetes RS-Flipflop}%
    \label{fig:rsflipfloptakt}
  \endminipage
\end{figure}

Dabei müssen wir eine Nebenbedingung $R \wedge S = 0$ setzen - $R$ und $S$ dürfen niemals gleichzeitig $= 1$ sein. In der Realisierung, dargestellt in Abb. \ref{fig:rsflipflop}, führt dies zu oszillationen. 

Will man ein taktgesteuertes RS-Flipflop, so braucht man lediglich das Taktsignal mit einem UND-Gatter jeweils mit dem $R$- und $S$-Eingang zu verbinden (siehe Abb. \ref{fig:rsflipfloptakt}).






\section{Kapitel/Abteil aus anderer Datei hinzufügen}

Lorem ipsum dolor sit amet, consectetur adipiscing elit. Vivamus porta lectus nec ante convallis lacinia. Nunc lobortis eu lacus nec euismod. Mauris at dapibus leo. Lorem ipsum dolor sit amet, consectetur adipiscing elit. Phasellus egestas luctus sapien ac venenatis. Sed maximus pellentesque arcu non ultrices. Aliquam erat volutpat. Nam pharetra orci in sem consequat, a dictum metus eleifend. Morbi non odio libero. Morbi porttitor in purus quis commodo. Aliquam erat volutpat. Nunc vitae arcu tempus, aliquet elit sit amet, rutrum diam. Curabitur imperdiet elementum iaculis. Vestibulum suscipit interdum libero, id ultrices est semper vitae.

In sed lacus malesuada, euismod mauris sit amet, lobortis metus. Maecenas iaculis ac mauris quis semper. Donec eget eros scelerisque, pellentesque nisl non, fermentum velit. Aliquam pharetra dolor risus, at tristique urna suscipit ut. Ut sit amet placerat mi. Maecenas felis felis, pharetra in tellus eget, blandit luctus nunc. Nunc mattis tortor vel nibh sollicitudin, id pulvinar quam accumsan. Nullam id risus id ipsum sollicitudin molestie ut hendrerit augue. Sed sit amet pharetra est, at interdum enim. Phasellus eget arcu vitae mauris lacinia tristique eu at nisl. Nam elementum vel mauris non aliquet. Morbi vel felis lobortis, pulvinar urna ut, venenatis nulla. Aliquam euismod eleifend est, consequat laoreet lorem posuere et. Fusce lectus erat, dapibus rhoncus tincidunt vel, porta quis libero. 









\section{Text in Boxen}

\begin{mdframed}[style=exampledefault,frametitle={Braucht package mdframed.}]
Lorem ipsum dolor sit amet, consectetur adipiscing elit. Vivamus porta lectus nec ante convallis lacinia. Nunc lobortis eu lacus nec euismod. Mauris at dapibus leo. Lorem ipsum dolor sit amet, consectetur adipiscing elit. Phasellus egestas luctus sapien ac venenatis. Sed maximus pellentesque arcu non ultrices. Aliquam erat volutpat. Nam pharetra orci in sem consequat, a dictum metus eleifend. Morbi non odio libero. Morbi porttitor in purus quis commodo. Aliquam erat volutpat. Nunc vitae arcu tempus, aliquet elit sit amet, rutrum diam. Curabitur imperdiet elementum iaculis. Vestibulum suscipit interdum libero, id ultrices est semper vitae.
\end{mdframed}


\begin{mdframed}[style=simple,frametitle={Mein style.}]

In sed lacus malesuada, euismod mauris sit amet, lobortis metus. Maecenas iaculis ac mauris quis semper. Donec eget eros scelerisque, pellentesque nisl non, fermentum velit. Aliquam pharetra dolor risus, at tristique urna suscipit ut. Ut sit amet placerat mi. Maecenas felis felis, pharetra in tellus eget, blandit luctus nunc. Nunc mattis tortor vel nibh sollicitudin, id pulvinar quam accumsan. Nullam id risus id ipsum sollicitudin molestie ut hendrerit augue. Sed sit amet pharetra est, at interdum enim. Phasellus eget arcu vitae mauris lacinia tristique eu at nisl. Nam elementum vel mauris non aliquet. Morbi vel felis lobortis, pulvinar urna ut, venenatis nulla. Aliquam euismod eleifend est, consequat laoreet lorem posuere et. Fusce lectus erat, dapibus rhoncus tincidunt vel, porta quis libero. 

\end{mdframed}




\section{Bibliographie und Zitate}
Einstein said something important\cite{einstein} and so did Dirac \cite{dirac}.


So they both said something important \cite{dirac,einstein}.

Und hier ist die Referenz zum Glossareintrag von \gls{computer}.


\section{Mathematik und Symbole}
\subsection{Sonderzeichen und Symbole}

\begin{flalign}
	 & \equiv \ \ll \ \lll \ \gg \ \ggg \ \leq \ \geq \ \leqslant \ \geqslant \ \propto \ \approx \ \approxeq \ \neq \ \simeq \ \cong \ \ncong
\end{flalign}

\begin{flalign}
	 & \cdot \ \times \ \vee \ \wedge \ \veebar \ \barwedge \pm \ \mp \ \sqrt{a} \ \sqrt[3]{a} \ \langle \ \rangle \ \infty
\end{flalign}

\begin{flalign}
	 & \leftarrow \ \rightarrow \ \Leftarrow \ \Rightarrow \ \parallel \ \bot
\end{flalign}

\begin{flalign}
	 %
	 & \in \ \notin\ \forall \ \exists \ \nexists \ \ni \ \mathbb{RNZ} \ \subset \ \supset \ \subseteq \ \supseteq
\end{flalign}

\begin{flalign}
	 & \int\limits_{1}^{2} \ \oint \ \iint \ \iiint \ \prod \ \sum
\end{flalign}

\begin{flalign}
	 & \ \vec{r} \ \bar{r} \ \dot{r} \ \ddot{r} \ \mathbf{r} \ \underline{r}
\end{flalign}

\begin{flalign}
	 & \odot \ \nabla \ \partial \ \hbar \ \mathcal{O}
\end{flalign}


















\subsection{Gleichungen und Spezielles}
\begin{align}
	 %
	 \vec{S}_\mu &= \vec{S}_\mu ^{\parallel}(0) \vec{u} + \vec{S}_\mu^{\bot} (0) [\cos(\omega_\mu t)\vec{v} - \sin(\omega_\mu t)\vec{w}]
\end{align}


\begin{align}
	 \vec{P} &= \frac{2}{\hbar} \langle \Psi | \hat{S} | \Psi \rangle = \vec{S}
\end{align}


\begin{align}
	 %
	 % UNDERBRACE
	 \Rightarrow \varphi (x) &= \sum\limits_{L = 0}^{\infty} \sum\limits_{m = - L}^{L} \sqrt{\frac{4 \pi}{2 L + 1}}  \underbrace{\int\limits_{\mathbb{R}^3} \sqrt{\frac{4 \pi}{2 L + 1}} \rho(\vec{x} ' ) r'^{L} Y_{l, m}^{*} (\theta ', \varphi ') d^3 x'}_{q_{l,m}}  \frac{Y_{l,m}(\theta, \varphi)}{r^ {L + 1}} \\ 
	 %
	 &= \sum\limits_{L = 0}^{\infty} \sum\limits_{m = - L}^{L} \sqrt{\frac{4 \pi}{2 L + 1}} q_{l,m}  \frac{Y_{l,m}(\theta, \varphi)}{r^ {L + 1}}
\end{align}


\begin{align}
	 % CASES
	 \nonumber
	 \Rightarrow q_{0, 0} &= \int \limits_{\mathbb{R}^3} \rho(\vec{x}' ) d^3 x' \entspricht 
	 \begin{cases} 
	 \text{total charge (electrostatics)} \\
	 \text{total mass (gravitation)}\\
	 \end{cases}
\end{align}


\begin{align}
	 % 3dim-VECTOR
	 \vec{r} &= \left( \begin{matrix}
	 r \cos \varphi \sin \theta\\
	 r \sin \varphi \sin \theta\\
	 r \cos \varphi
	 \end{matrix} \right )
\end{align}


\begin{align}
	 %
	 ... &=\begin{vmatrix}
		  -1-\lambda & 1 & -1 \\
		  2 & -1-\lambda & 2\\
		  2 & 2 & -1-\lambda\\
		  \end{vmatrix}
\end{align}


\begin{align}
	 \begin{pmatrix}
	 0\\0\\0
	 \end{pmatrix} &= \begin{pmatrix}
	 0 & 1 & -1 \\
	 2 & 0 & 2\\
	 2 & 2 & 0\\
	 \end{pmatrix} \begin{pmatrix}
	 x_1 \\ y_1\\ z_1
	 \end{pmatrix}
\end{align}


\begin{align}
	 %
	 z + \bar{z} &\leq 2\sqrt{z\bar{z}} \tag*{:2} \\\nonumber	 
	 %
	 Re(z) &\leq |z| = \sqrt{Re(z)^2 + Im(z)^2} \tag*{$\square$}
\end{align}


\begin{align}
	 |\sin z| &\overset{3b)}= \sqrt{\sin^2 x}
\end{align}


\begin{align}
	 \cosh(y) & \overset{y \in \mathbb{R}} \geq 1 \Rightarrow x = n \pi, n \in \mathbb{Z}
\end{align}
	
	
\begin{align}
	 f(z) &= \lim\limits_{x\rightarrow \infty} \frac{\sin x}{x} = 0
\end{align}


\begin{align}
	 f^{(n)} (z_0) &= \frac{n!}{2\pi i} \oint_C  \frac{f(z)}{(z - z_0)^{n + 1}}
\end{align}


\begin{align}
	 \binom{a}{n} &= \frac{a!}{(a-n)! n!}
\end{align}


\begin{align}
	 \nonumber \limz{\epsilon}\int(z) dz &= \limz{\epsilon} \frac14 \left[ \int\frac{e^{ia(u + 1)}}{u} du - \int\frac{e^{ia(u + 1)}}{u + 2}du   \right]\\[1em]
	 & \overset{z = 1 \Rightarrow u = 0}= \limz{\epsilon} \frac{e^{i a}}{4} \left[\vphantom{ \int\limits_{\pi}^0} \smash{ \underbrace{\frac{\overbrace{e^{ia \epsilon e^{i \varphi}}}^{\rightarrow 1}} {\epsilon e^{i \varphi}} i \epsilon e^{i \varphi}}_{\rightarrow i}  d \varphi            - \int\limits_{\pi}^0 \underbrace{\frac{\overbrace{e^{ia \epsilon e^{i \varphi}}}^{\rightarrow 1}} {\underbrace{\epsilon e^{i \varphi}}_{\rightarrow 0} + 2} \underbrace{i \epsilon e^{i \varphi}}_{\rightarrow 0}}_{\rightarrow 0}  d \varphi  }\right]\\[2em]
	 %	
	 %
	 2 + 2 &= 4 \ \text{some more space after this line please.}\\[4em]\nonumber
\end{align}


\begin{align}
	 \nonumber  2 + 2 &= 4 \text{ \hspace{1cm} unnumbered line.}\\
	 %
	 %
	 %
	 &\text{last line is made of text. Yay!}
\end{align}




In-line maths elements can be set with a different style: \(f(x) = \displaystyle \frac{1}{1+x}\). The same is true the other way around:

\begin{eqnarray*}
	f(x) = \sum_{i=0}^{n} \frac{a_i}{1+x} \\
	\textstyle f(x) = \textstyle \sum_{i=0}^{n} \frac{a_i}{1+x} \\
	\scriptstyle f(x) = \scriptstyle \sum_{i=0}^{n} \frac{a_i}{1+x} \\
	\scriptscriptstyle f(x) = \scriptscriptstyle \sum_{i=0}^{n} \frac{a_i}{1+x}
\end{eqnarray*}
















%########################################################################
%########################################################################
%########################################################################
\clearpage
\begin{appendix}
\setcounter{figure}{0} %Bildnummerierung auf Null setzen
\renewcommand{\thefigure}{A\arabic{figure}} %Bilder mit A$Bildnummer beschriften
\section{Bildanhang}


\begin{figure}[h!]
\centering
\minipage{0.4\textwidth}
  \fbox{\includegraphics[height = 3cm, keepaspectratio,draft]{images/random_image.png}}%
  \caption{Draft option \vspace{27pt}}%
  \label{fig:jkflipflop}
\endminipage\hspace{.5cm}
%
\minipage{0.4\textwidth}
  \fbox{\includegraphics[height = 3cm, keepaspectratio]{images/random_image.png}}%
  \caption{JK-Flipflop, Darstellung mit RS-Flipflop. C = Takt, $Q_1 = Q$, $Q_2 = \bar{Q}$}%
  \label{fig:JKmitRS}
\endminipage
\end{figure}

\section{Zeilenumbruch}
asdfghjklösdfghjklösdfghjklöasdfghjklösdfghjklösdfghj111klöasdfghjklösdfghjklösdfghjklöasd222\-fghjklösdfghjklösdfghjklöasdfghjklösdfghjklösdfghjklöasdfghjklösdfghjklö


\section{Neue Befehle definieren}
\newcommand{\musr}{$\mu$SR }
\musr


\newcommand{\myint}{\int\limits_{-\infty}^{\infty} dr \int\limits_{0}^{2 \pi} \sin(\vartheta) \varepsilon d\varphi}

\newcommand{\mysum}[3]{\sum\limits_{j = 0}^{\infty} \frac{#1\cdot #2}{#3} e^{- 3 \cos(\theta \phi)}}

\begin{align*}
  &\text{myint} & \myint\\
  &\text{mysum} & \mysum{x}{y}{z}
\end{align*}

\renewcommand{\myint}{myint ist jetzt ein neuer Befehl und macht nur noch das hier.}

\myint

\vspace{2cm}
\newcommand{\fett}[1]{{\bf #1}}

\fett{fettes hallo}

\newcommand{\kursiv}[1]{{\it #1}}

\kursiv{kursives hallo}



\end{appendix}





%------------
%Bibliography
%------------
%https://www.sharelatex.com/learn/Bibliography_management_in_LaTeX
%\medskip % or: \bigskip to add some space


\printbibliography
%\printbibliography[heading=bibintoc,title={Set some title instead of "References"}]
%if heading=bibintoc: set references in table of contents









%------------
%Glossary
%------------


\newglossaryentry{computer}
{
	name=computer,
	description={is a programmable machine that receives input,
		stores and manipulates data, and provides
		output in a useful format}
}


\newglossaryentry{real number}
{
	name={real number},
	description={include both rational numbers, such as $42$ and 
		$\frac{-23}{129}$, and irrational numbers, 
		such as $\pi$ and the square root of two; or,
		a real number can be given by an infinite decimal
		representation, such as $2.4871773339\ldots$ where
		the digits continue in some way; or, the real
		numbers may be thought of as points on an infinitely
		long number line.},
	symbol={\ensuremath{\mathbb{R}}}
}


%\glsaddall
\printglossary[title=My Glossary, toctitle=Glossary Title in ToC ]


\end{document}
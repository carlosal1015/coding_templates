\documentclass[12pt, a4paper, twopage]{scrartcl}

%\documentclass[
%11pt, 				% The default document font size, options: 10pt, 11pt, 12pt
%oneside, 			% Two side (alternating margins) for binding by default, uncomment to switch to one side
%chapterinoneline,	% Have the chapter title next to the number in one single line
%english, 			% ngerman for German
%singlespacing, 	% Single line spacing, alternatives: onehalfspacing or doublespacing
%draft, 			% Uncomment to enable draft mode (no pictures, no links, overfull hboxes indicated)
%nolistspacing, 	% If the document is onehalfspacing or doublespacing, uncomment this to set spacing in lists to single
%liststotoc, 		% Uncomment to add the list of figures/tables/etc to the table of contents
%toctotoc, 			% Uncomment to add the main table of contents to the table of contents
%parskip, 			% Uncomment to add space between paragraphs
%nohyperref, 		% Uncomment to not load the hyperref package
%headsepline, 		% Uncomment to get a line under the header
%]{scrartcl or scrreprt or scrbook} % The class file specifying the document structure


%-----------------------------------------------
% LOADING PACKAGES
%-----------------------------------------------

\usepackage{lmodern} 		% Diese beiden packages sorgen für echte 
\usepackage[T1]{fontenc}	% Umlaute.


\usepackage{amssymb, amsmath, color, dsfont, graphicx, float, setspace, tipa}
\usepackage[utf8]{inputenc} 
\usepackage[english]{babel}
\usepackage[pdfpagelabels,
			pdfstartview = FitH,
			bookmarksopen = true,
			bookmarksnumbered = true,
			linkcolor = black,
			plainpages = false,
			hypertexnames = false,
			citecolor = black,
			breaklinks]{hyperref}
\usepackage{url}
\usepackage{longtable} 		% Tables along multiple pages
\usepackage{caption}
\captionsetup{font=small,labelfont=bf, format=plain, justification=centering}
\allowdisplaybreaks 		% allows page breaks in align/equation environment

\usepackage{authblk} 		% titlepage stuff
\usepackage[titletoc, title]{appendix}

\usepackage{newclude} 		% use \include*{file} instead \include{} to omit pagebreak after include




%===================
% BIBLIOGRAPHY
%===================
\usepackage[backend=bibtex,
			sorting=nyt,
			bibencoding=ascii,
			citestyle=authoryear]{biblatex}
\addbibresource{references.bib}
\usepackage{csquotes} 		%recommended when using babel and biblatex
%DONT FORGET TO COMPILE THE BIBLIOGRAPHY WITH BIBTEX WHEN CHANGES ARE MADE.



%===================
% glossary
%===================
%\usepackage[toc,acronyms]{glossaries} % glossary. Set after usepackage{hyperref}
\usepackage{glossaries} % glossary. Set after usepackage{hyperref}
\makeglossaries

% IMPORTANT NOTE:
% Don't forget to recompile the glossary and the bibliography first if you want them to show up!

% Pseudocode
\usepackage{algorithm}
\usepackage[noend]{algpseudocode}
\makeatletter
\def\BState{\State\hskip-\ALG@thistlm}
\makeatother







%-----------------------------------------------
% TEXT IN BOXES
%-----------------------------------------------

\usepackage[framemethod=TikZ]{mdframed}
\mdfdefinestyle{exampledefault}{%
	rightline=true,innerleftmargin=10,innerrightmargin=10,
	frametitlerule=true,frametitlerulecolor=green,
	frametitlebackgroundcolor=yellow,
	frametitlerulewidth=2pt}
\mdfdefinestyle{simple}{%
	%rightline=true,
	innerleftmargin=10,
	innerrightmargin=10,
	%frametitlerule=true,
	%frametitlerulecolor=black,
	frametitlebackgroundcolor=lightgray,
	frametitlerulewidth=2
	}






%--------------------------------------------
% New Commands
%--------------------------------------------

\newcommand{\corresponds}{\mathrel{\widehat{=}}}       % equals with hat

\newcommand {\arctanh}{\mathrm{arctanh}}               % Atanh
\newcommand{\arccot}{\mathrm{arccot }}                 % Acotanh

\newcommand{\limz}[1]{\lim\limits_{#1 \rightarrow 0}}  % Limes of something towards zero

\newcommand{\bm}{\boldmath}                            % Bold font in math
\newcommand{\dps}{\displaystyle}                                               

\newcommand{\e}{\mbox{e}}                              % e noncursive in math mode

\newcommand{\del}{\partial}                            % partial diff operator
\newcommand{\de}{\mathrm{d}}                           % differential d
\newcommand{\D}{\mathrm{d}}                            % differential d
\newcommand{\GRAD}{\mathrm{grad}\ }                    % gradient
\newcommand{\DIV}{\mathrm{div}\ }                      % divergence
\newcommand{\ROT}{\mathrm{rot}\ }                      % rotation

\newcommand{\CONST}{\mathrm{const.\ }}                 % constant
\newcommand{\var}{\mathrm{var}}                        % variance

\newcommand{\g}{^\circ}                                % degrees
\newcommand{\degr}{^\circ}                             % degrees

\newcommand{\msol}{M_\odot}                            % solar mass


\newcommand{\x}{\mathbf{x}}                            % x vector
\newcommand{\xdot}{\dot{\mathbf{x}}}                   % x dot vector
\newcommand{\xddot}{\ddot{\mathbf{x}}}                 % x doubledot vector
\newcommand{\R}{\mathbf{r}}                            % r vector
\newcommand{\rdot}{\dot{\mathbf{r}}}                   % r dot vector
\newcommand{\rddot}{\ddot{\mathbf{r}}}                 % r doubledot vector
\newcommand{\vel}{\mathbf{v}}                          % v vector
\newcommand{\V}{\mathbf{v}}                            % v vector
\newcommand{\vdot}{\dot{\mathbf{v}}}                   % v dot vector
\newcommand{\vddot}{\ddot{\mathbf{v}}}                 % v doubledot vector

\newcommand{\dete}{\mathrm{d}t}                        % dt
\newcommand{\delte}{\del t}                            % partial t
\newcommand{\dex}{\mathrm{d}x}                         % dx
\newcommand{\delx}{\del x}                             % partial x
\newcommand{\der}{\mathrm{d}r}                         % dr
\newcommand{\delr}{\del r}                             % partial r







%-----------------------------------------------
% FORMAT TITLE
%-----------------------------------------------


% Set fonts of document parts
\setkomafont{title}{\rmfamily\bfseries\boldmath}
\addtokomafont{section}{\rmfamily\bfseries\boldmath}
\addtokomafont{subsection}{\rmfamily\bfseries\boldmath}
\addtokomafont{subsubsection}{\rmfamily\bfseries\boldmath}
\addtokomafont{disposition}{\rmfamily} % table of contents and stuff
\setkomafont{descriptionlabel}{\rmfamily\bfseries\boldmath}





%--------------------------------------------
%     OPTIONAL
%--------------------------------------------

%% Change font
%\newcommand{\changefont}[3]{
%\fontfamily{#1} \fontseries{#2} \fontshape{#3} \selectfont}
%\changefont{ppl}{m}{n} nach \begin{document} einsetzen

%% Fig. instead of Figure, Tab. instead of Table
%\usepackage[footnotesize]{caption2}
\addto\captionsenglish{\renewcommand{\figurename}{Fig.}}
%\addto\captionsngerman{\renewcommand{\figurename}{Fig.}}
%\renewcommand{\tablename}{Tab.}%


%\pagestyle{headings} % Write headings on each page

%\usepackage{chngcntr} \counterwithout{figure}{section} % Integer only figure numbers, ignoring chapter numbers






%------------------------------
% Set paper margins
%------------------------------

%\usepackage{geometry}
%\geometry{
%	a4paper,
%	total={170mm,257mm},
%	left=20mm,
%	top=20mm,
%}










%------------------------------------------
%:Metadata

\title{Title}
\author{Mladen Ivkovic\\
mladen.ivkovic@hotmail.com\\
}
\date{Date}

%------------------------------------------










%============================================================================================
%============================================================================================










\begin{document}
		
		
%--------------------------------------------
% Stuff that needs to be done before all else
%--------------------------------------------
%\pagestyle{plain}
\nocite{*} % show all entries of bibliography, even if they are not cited.
		
		
%\pagestyle{plain}





%Titlepage

\maketitle
\clearpage

%Table of Contents
\tableofcontents %Auf englisch wechseln: Ändere usepackage ngeman babel in english babel
\clearpage







%=============================================
\section*{Preamble}
%=============================================

This section is not in the table of contents and is not enumerated.


\begin{description}
  \item[Zweck] Dieses Dokument blablabla.
  \item[Punkt 2] Punkt 2
\end{description}


\clearpage




















%====================================
\section{Section}
%====================================





%====================================
\subsection{Subsection}
%====================================






%====================================
\subsubsection{Subsubsection}
%====================================

%

\begin{figure}
	\centering
	\fbox{
		\includegraphics[width = 3.5cm, keepaspectratio]{figures/random_image.png}
	}
	\caption{Figure caption goes here\label{fig:tabelle_zweierkomplement}}
\end{figure}
%
Die gängigste Form der Zahlensysteme sind Stellenwertsysteme. Eine Zahl $a$ wird in Form einer Reihe von Ziffern $z_i$ mit dazugehöriger Potenz der Basis $b^i$ dargestellt. Der Wert der Zahl ergibt sich dann als Summe der Werte aller Einzelstellen: $a = \sum\limits_{i}z_ib^i$.

\textbf{Umrechnung} in andere Zahlensysteme: Gegeben sei Zahl $Z$, umzuwandeln in System mit Basis $b$.
Eine angenehme Vorgehensweise gibt uns das \textbf{Horner Schema}\footnote{
Website mit Umrechnungen und Erklärungen: \url{http://www.arndt-bruenner.de/mathe/scripts/Zahlensysteme.htm}
}: Dividiere $Z$ durch $b$. Der Rest dieser Division ist die letzte Stelle der Zahl in der Basis $b$  (Einerstelle). Dividiere den Quotienten dieser Division wieder durch $b$. Der Rest dieser zweiten Division ergibt die zweite Stelle der Zahl in der neuen Basis. Wiederhole Divisionen, bis kein Rest mehr.







%====================================
\section{Tables}
%====================================



%====================================
\subsection{Simple}
%====================================
\begin{center}
\begin{tabular}[c]{c | c | c || c| c | c || c | c || c | c | c || c| c| c}
\multicolumn{3}{c||}{Konjunktion}	&	\multicolumn{3}{c||}{Disjunktion} & \multicolumn{2}{c||}{Negation} & \multicolumn{3}{c||}{NAND} & \multicolumn{3}{c}{NOR}\\
\multicolumn{3}{c||}{UND}	&	\multicolumn{3}{c||}{ODER} & \multicolumn{2}{c||}{} & \multicolumn{3}{c||}{} & \multicolumn{3}{c}{}\\
\hline
$a$ & $b$ & $a$ $\wedge$ $b$ & $a$ & $b$ & $a$ $\vee$ $b$ & $a$ & $\bar{a}$ & $a$ & $b$ & $\overline{a \wedge b}$ & $a$ & $b$ & $\overline{a \vee b}$\\
\hline
0 & 0 & 0 & 0 & 0 & 0 & 0 & 1 & 0 & 0 & 1 & 0 & 0 & 1\\
0 & 1 & 0 & 0 & 1 & 1 & 1 & 0 & 0 & 1 & 1 & 0 & 1 & 0\\
1 & 0 & 0 & 1 & 0 & 1 & & & 1 & 0 & 1 & 1 & 0 & 0\\
1 & 1 & 1 & 1 & 1 & 1 & & & 1 & 1 & 0 & 1 & 1 & 0\\
\hline
\end{tabular}
\end{center}








%===========================================
\subsection{With Extras and Pagebreaks}
%===========================================
\begin{spacing}{1.4}
	\begin{longtable}{p{4cm} l l}
		Kommutativgesetz: 	& $a \wedge b = b \wedge a$ & 
			$a \vee b = b \vee a$\\ 
		Distributivgesetz: 	& $a \wedge (b \vee c) = (a \wedge b) \vee (a \wedge b)$ & 
			$a \vee (b \wedge c) = (a \vee b) \wedge (a \vee c)$\\ 
		Neutrales Element 	& $a \wedge 1 = a$ & 
			$a \vee 0 = a$\\
		Inverses Element 	& $a \wedge \bar{a} = 0$ & 
			$a \vee \bar{a} = 1$\\
		Assoziativgesetz 	& $(a \wedge b) \wedge c = a \wedge (b \wedge c)$ & 
			$(a \vee b) \vee c = a \vee (b \vee c)$\\
		Idempotenzgesetz 	& $a \wedge a = a$ & 
			$a \vee a = a$\\
		Absorptionsgesetz 	& $a \wedge ( a \vee b) = a$ & 
			$a \vee (a \wedge b) = a$\\
		DeMorgan-Gesetz 	& $\overline{a\wedge b} = \bar{a} \vee \bar{b}$ (NAND)& 
			$\overline{a \vee b} = \bar{a} \wedge \bar{b}$ (NOR)\\
		\smash{Gesetz vom Widerspruch} & & 
			$a \wedge \overline{a} = 0$\\
		\smash{Gesetz vom ausgeschl. Dritten} & & 
			$a \vee \overline{a} = 1$ \\
		\smash{Gesetz der doppelten Negation} & & 
			$\overline{\overline{a}} = a$ \\
	\end{longtable}
\end{spacing}














%=============================================
\section{Two images, columns}
%=============================================

\begin{figure}[h!]
\centering
	\minipage{0.3\textwidth}
		\centering
		\fbox{\includegraphics[height=2.5cm, keepaspectratio]{figures/random_image.png}}%
		\caption{RS-Flipflop}%
		\label{fig:rsflipflop}
	\endminipage\hspace{1cm}   
%
	\minipage{0.4\textwidth}
		\centering
		\fbox{\includegraphics[height=2.5cm, keepaspectratio]{figures/random_image_2.png}}%
		\caption{getaktetes RS-Flipflop}%
		\label{fig:rsflipfloptakt}
	\endminipage
\end{figure}


Dabei müssen wir eine Nebenbedingung $R \wedge S = 0$ setzen - $R$ und $S$ dürfen niemals gleichzeitig $= 1$ sein. In der Realisierung, dargestellt in Abb. \ref{fig:rsflipflop}, führt dies zu oszillationen. 

Will man ein taktgesteuertes RS-Flipflop, so braucht man lediglich das Taktsignal mit einem UND-Gatter jeweils mit dem $R$- und $S$-Eingang zu verbinden (siehe Abb. \ref{fig:rsflipfloptakt}).\\



% first column
\begin{minipage}[t]{0.5\textwidth}
	This is the first column.\\
	
	This is still in the first column.
	\begin{align}
		a+b &= c\\
		d+e &= e
	\end{align}
\end{minipage}
%
%No newline here!
%
%second column
\begin{minipage}[t]{0.5\textwidth}
	This is the second column.
	
	Blablabla.
	\begin{align*}
		a+b &= c\\
		d+e &= e
	\end{align*}
\end{minipage}










%=============================================
%\section{Including files}
%=============================================
\section{Kapitel/Abteil aus anderer Datei hinzufügen}

Lorem ipsum dolor sit amet, consectetur adipiscing elit. Vivamus porta lectus nec ante convallis lacinia. Nunc lobortis eu lacus nec euismod. Mauris at dapibus leo. Lorem ipsum dolor sit amet, consectetur adipiscing elit. Phasellus egestas luctus sapien ac venenatis. Sed maximus pellentesque arcu non ultrices. Aliquam erat volutpat. Nam pharetra orci in sem consequat, a dictum metus eleifend. Morbi non odio libero. Morbi porttitor in purus quis commodo. Aliquam erat volutpat. Nunc vitae arcu tempus, aliquet elit sit amet, rutrum diam. Curabitur imperdiet elementum iaculis. Vestibulum suscipit interdum libero, id ultrices est semper vitae.

In sed lacus malesuada, euismod mauris sit amet, lobortis metus. Maecenas iaculis ac mauris quis semper. Donec eget eros scelerisque, pellentesque nisl non, fermentum velit. Aliquam pharetra dolor risus, at tristique urna suscipit ut. Ut sit amet placerat mi. Maecenas felis felis, pharetra in tellus eget, blandit luctus nunc. Nunc mattis tortor vel nibh sollicitudin, id pulvinar quam accumsan. Nullam id risus id ipsum sollicitudin molestie ut hendrerit augue. Sed sit amet pharetra est, at interdum enim. Phasellus eget arcu vitae mauris lacinia tristique eu at nisl. Nam elementum vel mauris non aliquet. Morbi vel felis lobortis, pulvinar urna ut, venenatis nulla. Aliquam euismod eleifend est, consequat laoreet lorem posuere et. Fusce lectus erat, dapibus rhoncus tincidunt vel, porta quis libero. 












%=============================================
\section{Text in Boxes}
%=============================================

\begin{mdframed}[style=exampledefault,frametitle={needs package mdframed.}]
	
	Lorem ipsum dolor sit amet, consectetur adipiscing elit. Vivamus porta lectus nec ante convallis lacinia. Nunc lobortis eu lacus nec euismod. Mauris at dapibus leo. Lorem ipsum dolor sit amet, consectetur adipiscing elit. Phasellus egestas luctus sapien ac venenatis. Sed maximus pellentesque arcu non ultrices. Aliquam erat volutpat. Nam pharetra orci in sem consequat, a dictum metus eleifend. Morbi non odio libero. Morbi porttitor in purus quis commodo. Aliquam erat volutpat. Nunc vitae arcu tempus, aliquet elit sit amet, rutrum diam. Curabitur imperdiet elementum iaculis. Vestibulum suscipit interdum libero, id ultrices est semper vitae.
	
\end{mdframed}


\begin{mdframed}[style=simple,frametitle={My style.}]

	In sed lacus malesuada, euismod mauris sit amet, lobortis metus. Maecenas iaculis ac mauris quis semper. Donec eget eros scelerisque, pellentesque nisl non, fermentum velit. Aliquam pharetra dolor risus, at tristique urna suscipit ut. Ut sit amet placerat mi. Maecenas felis felis, pharetra in tellus eget, blandit luctus nunc. Nunc mattis tortor vel nibh sollicitudin, id pulvinar quam accumsan. Nullam id risus id ipsum sollicitudin molestie ut hendrerit augue. Sed sit amet pharetra est, at interdum enim. Phasellus eget arcu vitae mauris lacinia tristique eu at nisl. Nam elementum vel mauris non aliquet. Morbi vel felis lobortis, pulvinar urna ut, venenatis nulla. Aliquam euismod eleifend est, consequat laoreet lorem posuere et. Fusce lectus erat, dapibus rhoncus tincidunt vel, porta quis libero. 

\end{mdframed}









%=============================================
\section{Bibliography and Citing}
%=============================================


Einstein said something important\cite{einstein} and so did Dirac \cite{dirac}.


So they both said something important \cite{dirac,einstein}.

And this is a reference to the glossary entry of \gls{computer}.















%=============================================
\section{Mathematics and Symbols}
%=============================================


%=============================================
\subsection{Special characters and symbols}
%=============================================

\begin{flalign}
	 & \equiv \ \ll \ \lll \ \gg \ \ggg \ \leq \ \geq \ \leqslant \ \geqslant \ \propto \ \approx \ \approxeq \ \neq \ \simeq \ \cong \ \ncong \ \overset{\widehat{=}} \ \overset{!}{=}
\end{flalign}

\begin{flalign}
	 & \cdot \ \times \ \vee \ \wedge \ \veebar \ \barwedge \pm \ \mp \ \sqrt{a} \ \sqrt[3]{a} \ \langle \ \rangle \ \infty
\end{flalign}

\begin{flalign}
	 & \leftarrow \ \rightarrow \ \Leftarrow \ \Rightarrow \ \parallel \ \bot
\end{flalign}

\begin{flalign}
	 %
	 & \in \ \notin\ \forall \ \exists \ \nexists \ \ni \ \subset \ \supset \ \subseteq \ \supseteq \
\end{flalign}
	 
\begin{align}
	& \mathcal{H \ L \ S \ T \ Q \ O}  \ \mathbb{R\ N \ Z}
\end{align}

\begin{align}
	& \mathds{R\ N \ Z\ 1\ 2\ 3\ C} % NEEDS DSFONT PACKAGE
\end{align}


\begin{flalign}
	 & \int\limits_{1}^{2} \ \oint \ \iint \ \iiint \ \prod \ \sum
\end{flalign}

\begin{flalign}
	 & \ \vec{r} \ \bar{r} \ \dot{r} \ \ddot{r} \ \mathbf{r} \ \underline{r}
\end{flalign}

\begin{flalign}
	 & \odot \ \nabla \ \partial \ \hbar \ \mathcal{O}
\end{flalign}

















%=============================================
\subsection{Equations and Special Stuff}
%=============================================
\begin{align}
	 \vec{S}_\mu &= \vec{S}_\mu ^{\parallel}(0) \vec{u} + \vec{S}_\mu^{\bot} (0) [\cos(\omega_\mu t)\vec{v} - \sin(\omega_\mu t)\vec{w}]
\end{align}

\begin{align}
	c_V = \frac{\mathrm{d} Q}{\mathrm{d} T} \bigg{\rvert}_V
\end{align}

\begin{align}
	 \vec{P} &= \frac{2}{\hbar} \langle \Psi | \hat{S} | \Psi \rangle = \vec{S}
\end{align}


\begin{align}
	 %
	 % UNDERBRACE
	 \Rightarrow \varphi (x) &= \sum\limits_{L = 0}^{\infty} \sum\limits_{m = - L}^{L} \sqrt{\frac{4 \pi}{2 L + 1}}  \underbrace{\int\limits_{\mathbb{R}^3} \sqrt{\frac{4 \pi}{2 L + 1}} \rho(\vec{x} ' ) r'^{L} Y_{l, m}^{*} (\theta ', \varphi ') d^3 x'}_{q_{l,m}}  \frac{Y_{l,m}(\theta, \varphi)}{r^ {L + 1}} \\ 
	 %
	 &= \sum\limits_{L = 0}^{\infty} \sum\limits_{m = - L}^{L} \sqrt{\frac{4 \pi}{2 L + 1}} q_{l,m}  \frac{Y_{l,m}(\theta, \varphi)}{r^ {L + 1}}
\end{align}


\begin{align}
	 % CASES
	 \nonumber
	 \Rightarrow q_{0, 0} &= \int \limits_{\mathbb{R}^3} \rho(\vec{x}' ) d^3 x' \corresponds 
	 \begin{cases} 
	 \text{total charge (electrostatics)} \\
	 \text{total mass (gravitation)}\\
	 \end{cases}
\end{align}


\begin{align}
	 % 3dim-VECTOR
	 \vec{r} &= \left( \begin{matrix}
	 r \cos \varphi \sin \theta\\
	 r \sin \varphi \sin \theta\\
	 r \cos \varphi
	 \end{matrix} \right )
\end{align}


\begin{align}
	 %
	 ... &=\begin{vmatrix}
		  -1-\lambda & 1 & -1 \\
		  2 & -1-\lambda & 2\\
		  2 & 2 & -1-\lambda\\
		  \end{vmatrix}
\end{align}


\begin{align}
	 \begin{pmatrix}
	 	0\\0\\0
	 \end{pmatrix} &= 
	 \begin{pmatrix}
		 0 & 1 & -1 \\
		 2 & 0 & 2\\
		 2 & 2 & 0\\
	 \end{pmatrix} 
	 \begin{pmatrix}
		 x_1 \\ y_1\\ z_1
	 \end{pmatrix}
\end{align}


\begin{align}
	 %
	 z + \bar{z} &\leq 2\sqrt{z\bar{z}} \tag*{:2} \\\nonumber	 
	 %
	 Re(z) &\leq |z| = \sqrt{Re(z)^2 + Im(z)^2} \tag*{$\square$}
\end{align}


\begin{align}
	 |\sin z| &\overset{3b)}= \sqrt{\sin^2 x}
\end{align}


\begin{align}
	 \cosh(y) & \overset{y \in \mathbb{R}} \geq 1 \Rightarrow x = n \pi, n \in \mathbb{Z}
\end{align}
	
	
\begin{align}
	 f(z) &= \lim\limits_{x\rightarrow \infty} \frac{\sin x}{x} = 0
\end{align}


\begin{align}
	 f^{(n)} (z_0) &= \frac{n!}{2\pi i} \oint_C  \frac{f(z)}{(z - z_0)^{n + 1}}
\end{align}


\begin{align}
	 \binom{a}{n} &= \frac{a!}{(a-n)! n!}
\end{align}


\begin{align}
	 \nonumber \limz{\epsilon}\int(z) dz &= 
	 	\limz{\epsilon} \frac14 \left[ \int\frac{e^{ia(u + 1)}}{u} du - 
	 	\int\frac{e^{ia(u + 1)}}{u + 2}du   \right]\\[1em]
	 & \overset{z = 1 \Rightarrow u = 0} = 
	 	\limz{\epsilon} \frac{e^{i a}}{4} 
	 	\left[
	 		\vphantom{ \int\limits_{\pi}^0} 
		 	\smash{ 
		 		\underbrace{
		 			\frac{\overbrace{
		 					e^{ia \epsilon e^{i \varphi}}
	 					}^{\rightarrow 1}
 					} {
 						\epsilon e^{i \varphi}
 					} 
 					i \epsilon e^{i \varphi}
	 			}_{\rightarrow i}  
 				d \varphi   -
				\int\limits_{\pi}^0 \underbrace{
					\frac{ 
						\overbrace{
							e^{ia \epsilon e^{i \varphi}}
						}^{\rightarrow 1}
					} {
					\underbrace{
						\epsilon e^{i \varphi}
					}_{\rightarrow 0} + 2} 
				\underbrace{i \epsilon e^{i \varphi}}_{\rightarrow 0}
				}_{\rightarrow 0}  
				d \varphi  
			}
		\right]\\[2em]
	 %	
	 %
	 2 + 2 &= 4 \ \text{some more space after this line please.}\\[4em]\nonumber
\end{align}


\begin{align}
	 \nonumber  2 + 2 &= 4 \text{ \hspace{1cm} unnumbered line.}\\
	 %
	 %
	 %
	 &\text{last line is made of text. Yay!}
\end{align}




In-line maths elements can be set with a different style: \(f(x) = \displaystyle \frac{1}{1+x}\). The same is true the other way around:

\begin{eqnarray*}
	f(x) = \sum_{i=0}^{n} \frac{a_i}{1+x} \\
	\textstyle f(x) = \textstyle \sum_{i=0}^{n} \frac{a_i}{1+x} \\
	\scriptstyle f(x) = \scriptstyle \sum_{i=0}^{n} \frac{a_i}{1+x} \\
	\scriptscriptstyle f(x) = \scriptscriptstyle \sum_{i=0}^{n} \frac{a_i}{1+x}
\end{eqnarray*}


Two columns: Version 1
%two columns
\begin{align}
	\begin{split}
		&\text{Column 1}\\
		a &= b+c\\
		a &= b+c\\
	\end{split}
	\begin{split}
		& \text{Column 2}\\
		2a &= b+c\\
		2a &= b+c\\
	\end{split}
\end{align}

Version 2:

% first column
\begin{minipage}[t]{0.4\textwidth}
	This is the first column.\\
	
	This is still in the first column.
	\begin{align}
	a+b &= c\\
	d+e &= e
	\end{align}
\end{minipage}
%
%No newline here!
%
%second column
\begin{minipage}[t]{0.4\textwidth}
	This is the second column.
	\begin{align}
	a+b &= c\\
	d+e &= e
	\end{align}
\end{minipage}













%============================================================================================
%============================================================================================




\clearpage

%=============================================
\section{Pseudocode}
%=============================================

%\usepackage{algorithm}
%\usepackage[noend]{algpseudocode}
%\makeatletter
%\def\BState{\State\hskip-\ALG@thistlm}
%\makeatother

%https://en.wikibooks.org/wiki/LaTeX/Algorithms

\begin{algorithm}
	\caption{My algorithm}\label{euclid}
	\begin{algorithmic}[1]
		\Procedure{Unbinding Routine}{}
		\State $\textit{stringlen} \gets \text{length of }\textit{string}$
		\State $i \gets \textit{patlen}$
		
		
		\BState \emph{top}:
		
		\If {$i > \textit{stringlen}$} \Return false
		\EndIf
		\State $j \gets \textit{patlen}$
		
		
		\BState \emph{other loop}:
		
		\While {$i < j$}:
		\State $i \gets i+1$
		\Comment This is a comment.
		\ForAll{elements $ \in $ list}
		\State Do something.
		\EndFor
		\EndWhile
		
		
		\BState \emph{functions:}
		\Function{myfunc}{x,y}
		\State x = y-2
		\Return x
		\EndFunction
		\EndProcedure
	\end{algorithmic}
\end{algorithm}





%============================================================================================
%============================================================================================




\clearpage


%=============================================
\begin{appendix}
%=============================================
\setcounter{figure}{0} % Set figure counter back to zero
\renewcommand{\thefigure}{A\arabic{figure}} % Set figure numbering to A<integer>






%=============================================
\section{Image appendix}
%=============================================


\begin{figure}[h!]
\centering
\minipage{0.4\textwidth}
  \fbox{\includegraphics[height = 3cm, keepaspectratio,draft]{figures/random_image.png}}%
  \caption{Draft option \vspace{27pt}}%
  \label{fig:jkflipflop}
\endminipage\hspace{.5cm}
%
\minipage{0.4\textwidth}
  \fbox{\includegraphics[height = 3cm, keepaspectratio]{figures/random_image.png}}%
  \caption{JK-Flipflop, Darstellung mit RS-Flipflop. C = Takt, $Q_1 = Q$, $Q_2 = \bar{Q}$}%
  \label{fig:JKmitRS}
\endminipage
\end{figure}









%=============================================
\section{Line break}
%=============================================
asdfghjklösdfghjklösdfghjklöasdfghjklösdfghjklösdfghj111klöasdfghjklösdfghjklösdfghjklöasd222\-fghjklösdfghjklösdfghjklöasdfghjklösdfghjklösdfghjklöasdfghjklösdfghjklö








%=============================================
\section{Defining new commands}
%=============================================
\newcommand{\musr}{$\mu$SR }
\musr


\newcommand{\myint}{\int\limits_{-\infty}^{\infty} dr \int\limits_{0}^{2 \pi} \sin(\vartheta) \varepsilon d\varphi}

\newcommand{\mysum}[3]{\sum\limits_{j = 0}^{\infty} \frac{#1\cdot #2}{#3} e^{- 3 \cos(\theta \phi)}}

\begin{align*}
  &\text{myint} & \myint\\
  &\text{mysum} & \mysum{x}{y}{z}
\end{align*}

\renewcommand{\myint}{myint is now a new command and does this:}

\myint

\vspace{2cm}
\newcommand{\fett}[1]{{\textbf{#1}}}

\fett{phat hello}

\newcommand{\kursiv}[1]{{\textit{#1}}}

\kursiv{cursive hallo}



\end{appendix}








%=============================================
% Bibliography
%=============================================
%https://www.sharelatex.com/learn/Bibliography_management_in_LaTeX
%\medskip % or: \bigskip to add some space


\printbibliography
%\printbibliography[heading=bibintoc,title={Set some title instead of "References"}]
%if heading=bibintoc: set references in table of contents









%=============================================
% Glossary
%=============================================


\newglossaryentry{computer}
{
	name=computer,
	description={is a programmable machine that receives input,
		stores and manipulates data, and provides
		output in a useful format}
}


\newglossaryentry{real number}
{
	name={real number},
	description={include both rational numbers, such as $42$ and 
		$\frac{-23}{129}$, and irrational numbers, 
		such as $\pi$ and the square root of two; or,
		a real number can be given by an infinite decimal
		representation, such as $2.4871773339\ldots$ where
		the digits continue in some way; or, the real
		numbers may be thought of as points on an infinitely
		long number line.},
	symbol={\ensuremath{\mathbb{R}}}
}


%\glsaddall
\printglossary[title=My Glossary, toctitle=Glossary Title in ToC ]


\end{document}
%==============================================
% This file contains my definitions and
% newcommands. Makes things easy to copypaste
% between projects.
%==============================================


%--------------------------------------------
% Math Stuff
%--------------------------------------------

\newcommand{\corresponds}{\mathrel{\widehat{=}}}       % equals with hat

\newcommand {\arctanh}{\mathrm{arctanh}}               % Atanh
\newcommand{\arccot}{\mathrm{arccot }}                 % Acotanh

\newcommand{\limz}[1]{\lim\limits_{#1 \rightarrow 0}}  % Limes of something towards zero

\newcommand{\bm}{\boldmath}                            % Bold font in math
\newcommand{\dps}{\displaystyle}                                               

\newcommand{\e}{\mbox{e}}                              % e noncursive in math mode

\newcommand{\del}{\partial}                            % partial diff operator
\newcommand{\de}{\mathrm{d}}                           % differential d
\newcommand{\D}{\mathrm{d}}                            % differential d
\newcommand{\GRAD}{\mathrm{grad}\ }                    % gradient
\newcommand{\DIV}{\mathrm{div}\ }                      % divergence
\newcommand{\ROT}{\mathrm{rot}\ }                      % rotation

\newcommand{\CONST}{\mathrm{const.\ }}                 % constant
\newcommand{\var}{\mathrm{var}}                        % variance

\newcommand{\g}{^\circ}                                % degrees
\newcommand{\degr}{^\circ}                             % degrees

\newcommand{\msol}{M_\odot}                            % solar mass
\newcommand{\order}{\mathcal{O}}                       % order, e.g. O(h^2)


\newcommand{\x}{\mathbf{x}}                            % x vector
\newcommand{\xdot}{\dot{\mathbf{x}}}                   % x dot vector
\newcommand{\xddot}{\ddot{\mathbf{x}}}                 % x doubledot vector
\newcommand{\R}{\mathbf{r}}                            % r vector
\newcommand{\rdot}{\dot{\mathbf{r}}}                   % r dot vector
\newcommand{\rddot}{\ddot{\mathbf{r}}}                 % r doubledot vector
\newcommand{\vel}{\mathbf{v}}                          % v vector
\newcommand{\V}{\mathbf{v}}                            % v vector
\newcommand{\vdot}{\dot{\mathbf{v}}}                   % v dot vector
\newcommand{\vddot}{\ddot{\mathbf{v}}}                 % v doubledot vector

\newcommand{\dete}{\mathrm{d}t}                        % dt
\newcommand{\delte}{\del t}                            % partial t
\newcommand{\dex}{\mathrm{d}x}                         % dx
\newcommand{\delx}{\del x}                             % partial x
\newcommand{\der}{\mathrm{d}r}                         % dr
\newcommand{\delr}{\del r}                             % partial r


\newcommand{\deldx}{\frac{\del}{\del x}}				% shortcut partial derivative, in line
\newcommand{\ddx}{\frac{\de}{\de x}}					% shortcut total derivative, in line
\newcommand{\DELDX}[1]{\frac{\del  #1}{\del x}}			% shortcut partial derivative, on fraction
\newcommand{\DDX}[1]{\frac{\de  #1}{\de x}}				% shortcut total derivative, on fraction

\newcommand{\deldvecx}{\frac{\del}{\del \x}}	   		% shortcut partial derivative, in line
\newcommand{\ddvecx}{\frac{\de}{\de \x}}				% shortcut total derivative, in line
\newcommand{\DELDVECX}[1]{\frac{\del  #1}{\del \x}}		% shortcut partial derivative, on fraction
\newcommand{\DDVECX}[1]{\frac{\de  #1}{\de \x}}			% shortcut total derivative, on fraction

\newcommand{\deldr}{\frac{\del}{\del r}}				% shortcut partial derivative, in line
\newcommand{\ddr}{\frac{\de}{\de r}}					% shortcut total derivative, in line
\newcommand{\DELDR}[1]{\frac{\del  #1}{\del r}}			% shortcut partial derivative, on fraction
\newcommand{\DDR}[1]{\frac{\de  #1}{\de r}}				% shortcut total derivative, on fraction

\newcommand{\deldt}{\frac{\del}{\del t}}				% shortcut partial derivative, in line
\newcommand{\ddt}{\frac{\de}{\de t}}					% shortcut total derivative, in line
\newcommand{\DELDT}[1]{\frac{\del  #1}{\del t}}			% shortcut partial derivative, on fraction
\newcommand{\DDT}[1]{\frac{\de  #1}{\de t}}				% shortcut total derivative, on fraction






%-----------------------------------------------
% Work related / project specific math stuff
%-----------------------------------------------

\newcommand{\Aij}{$\mathbf{A}_{ij}$}	% A_ij
\newcommand{\Aijm}{\mathbf{A}_{ij}}		% A_ij math
\newcommand{\U}{\mathbf{U}}				% State vector
\newcommand{\F}{\mathbf{F}}				% Flux tensor
\newcommand{\psitilde}{\tilde{\psi}}	% psi tilde
\newcommand{\half}{1/2}                 % 1/2









%----------------------------------
% Redefinitions
%----------------------------------


% replace \sum with \sum\limits
\let\oldsum\sum
\renewcommand{\sum}{\oldsum\limits}






%-----------------------------------
% Shortcuts
%-----------------------------------

% shortcut for TODO box
%		usage: \todo{Your text here}
\newcommand{\todo}[1]{\begin{mdframed}[style=todo,frametitle={TODO}] #1 \end{mdframed}}
\newcommand{\note}[1]{\begin{mdframed}[style=eyecatcher,frametitle={Note}] #1 \end{mdframed}}


% quickly insert a figure without a caption
% 		usage: \quickfig{filename}{label}
\newcommand{\quickfig}[2]{
       \begin{figure}[H]
               \includegraphics[width=\textwidth]{#1}
               \caption{\label{#2}}
       \end{figure}
}


\newcommand{\quickfigcap}[3]{
       \begin{figure}[H]
               \includegraphics[width=\textwidth]{#1}
               \caption{#3\label{#2}}
       \end{figure}
}









%-----------------------------------------------
% TEXT IN BOXES
%-----------------------------------------------
 
\usepackage[framemethod=TikZ]{mdframed}

% New Colors (needs to be after usepackage mdframed)
\definecolor{babyblueeyes}{rgb}{0.63, 0.79, 0.95}
\definecolor{ashgrey}{rgb}{0.7, 0.75, 0.71}
\definecolor{caribbeangreen}{rgb}{0.0, 0.8, 0.6}
\definecolor{bittersweet}{rgb}{1.0, 0.44, 0.37}

\mdfdefinestyle{todo}{%
       %rightline=true,
       innerleftmargin=10,
       innerrightmargin=10,
       %frametitlerule=true,
       %frametitlerulecolor=black,
       frametitlebackgroundcolor=babyblueeyes,
       frametitlerulewidth=2
}


\mdfdefinestyle{eyecatcher}{%
       %rightline=true,
       innerleftmargin=10,
       innerrightmargin=10,
       %frametitlerule=true,
       %frametitlerulecolor=black,
       frametitlebackgroundcolor=bittersweet,
       frametitlerulewidth=2
}









%---------------------------------------------------
% Journal abbreviations
%---------------------------------------------------

%  http://adsabs.harvard.edu/abs_doc/aas_macros.sty
%
%  These Macros are taken from the AAS TeX macro package version 5.2
%  and are compatible with the macros in the A&A document class
%  version 7.0
%  Include this file in your LaTeX source only if you are not using
%  the AAS TeX macro package or the A&A document class and need to
%  resolve the macro definitions in the TeX/BibTeX entries returned by
%  the ADS abstract service.
%
%  If you plan not to use this file to resolve the journal macros
%  rather than the whole AAS TeX macro package, you should save the
%  file as ``aas_macros.sty'' and then include it in your LaTeX paper
%  by using a construct such as:
%	\documentstyle[11pt,aas_macros]{article}
%
%  For more information on the AASTeX and A&A packages, please see:
%       http://journals.aas.org/authors/aastex.html	
%       ftp://ftp.edpsciences.org/pub/aa/readme.html
%  For more information about ADS abstract server, please see:
%       http://adsabs.harvard.edu/ads_abstracts.html
%

% Abbreviations for journals.  The object here is to provide authors
% with convenient shorthands for the most "popular" (often-cited)
% journals; the author can use these markup tags without being concerned
% about the exact form of the journal abbreviation, or its formatting.
% It is up to the keeper of the macros to make sure the macros expand
% to the proper text.  If macro package writers agree to all use the
% same TeX command name, authors only have to remember one thing, and
% the style file will take care of editorial preferences.  This also
% applies when a single journal decides to revamp its abbreviating
% scheme, as happened with the ApJ (Abt 1991).


\makeatletter

\let\jnl@style=\rm
\def\ref@jnl#1{{\jnl@style#1}}

\def\aj{\ref@jnl{AJ}}                   % Astronomical Journal
\def\actaa{\ref@jnl{Acta Astron.}}      % Acta Astronomica
\def\araa{\ref@jnl{ARA\&A}}             % Annual Review of Astron and Astrophys
\def\apj{\ref@jnl{ApJ}}                 % Astrophysical Journal
\def\apjl{\ref@jnl{ApJ}}                % Astrophysical Journal, Letters
\def\apjs{\ref@jnl{ApJS}}               % Astrophysical Journal, Supplement
\def\ao{\ref@jnl{Appl.~Opt.}}           % Applied Optics
\def\apss{\ref@jnl{Ap\&SS}}             % Astrophysics and Space Science
\def\aap{\ref@jnl{A\&A}}                % Astronomy and Astrophysics
\def\aapr{\ref@jnl{A\&A~Rev.}}          % Astronomy and Astrophysics Reviews
\def\aaps{\ref@jnl{A\&AS}}              % Astronomy and Astrophysics, Supplement
\def\azh{\ref@jnl{AZh}}                 % Astronomicheskii Zhurnal
\def\baas{\ref@jnl{BAAS}}               % Bulletin of the AAS
\def\bac{\ref@jnl{Bull. astr. Inst. Czechosl.}}
                % Bulletin of the Astronomical Institutes of Czechoslovakia 
\def\caa{\ref@jnl{Chinese Astron. Astrophys.}}
                % Chinese Astronomy and Astrophysics
\def\cjaa{\ref@jnl{Chinese J. Astron. Astrophys.}}
                % Chinese Journal of Astronomy and Astrophysics
\def\icarus{\ref@jnl{Icarus}}           % Icarus
\def\jcap{\ref@jnl{J. Cosmology Astropart. Phys.}}
                % Journal of Cosmology and Astroparticle Physics
\def\jrasc{\ref@jnl{JRASC}}             % Journal of the RAS of Canada
\def\memras{\ref@jnl{MmRAS}}            % Memoirs of the RAS
\def\mnras{\ref@jnl{MNRAS}}             % Monthly Notices of the RAS
\def\na{\ref@jnl{New A}}                % New Astronomy
\def\nar{\ref@jnl{New A Rev.}}          % New Astronomy Review
\def\pra{\ref@jnl{Phys.~Rev.~A}}        % Physical Review A: General Physics
\def\prb{\ref@jnl{Phys.~Rev.~B}}        % Physical Review B: Solid State
\def\prc{\ref@jnl{Phys.~Rev.~C}}        % Physical Review C
\def\prd{\ref@jnl{Phys.~Rev.~D}}        % Physical Review D
\def\pre{\ref@jnl{Phys.~Rev.~E}}        % Physical Review E
\def\prl{\ref@jnl{Phys.~Rev.~Lett.}}    % Physical Review Letters
\def\pasa{\ref@jnl{PASA}}               % Publications of the Astron. Soc. of Australia
\def\pasp{\ref@jnl{PASP}}               % Publications of the ASP
\def\pasj{\ref@jnl{PASJ}}               % Publications of the ASJ
\def\rmxaa{\ref@jnl{Rev. Mexicana Astron. Astrofis.}}%
                % Revista Mexicana de Astronomia y Astrofisica
\def\qjras{\ref@jnl{QJRAS}}             % Quarterly Journal of the RAS
\def\skytel{\ref@jnl{S\&T}}             % Sky and Telescope
\def\solphys{\ref@jnl{Sol.~Phys.}}      % Solar Physics
\def\sovast{\ref@jnl{Soviet~Ast.}}      % Soviet Astronomy
\def\ssr{\ref@jnl{Space~Sci.~Rev.}}     % Space Science Reviews
\def\zap{\ref@jnl{ZAp}}                 % Zeitschrift fuer Astrophysik
\def\nat{\ref@jnl{Nature}}              % Nature
\def\iaucirc{\ref@jnl{IAU~Circ.}}       % IAU Cirulars
\def\aplett{\ref@jnl{Astrophys.~Lett.}} % Astrophysics Letters
\def\apspr{\ref@jnl{Astrophys.~Space~Phys.~Res.}}
                % Astrophysics Space Physics Research
\def\bain{\ref@jnl{Bull.~Astron.~Inst.~Netherlands}} 
                % Bulletin Astronomical Institute of the Netherlands
\def\fcp{\ref@jnl{Fund.~Cosmic~Phys.}}  % Fundamental Cosmic Physics
\def\gca{\ref@jnl{Geochim.~Cosmochim.~Acta}}   % Geochimica Cosmochimica Acta
\def\grl{\ref@jnl{Geophys.~Res.~Lett.}} % Geophysics Research Letters
\def\jcp{\ref@jnl{J.~Chem.~Phys.}}      % Journal of Chemical Physics
\def\jgr{\ref@jnl{J.~Geophys.~Res.}}    % Journal of Geophysics Research
\def\jqsrt{\ref@jnl{J.~Quant.~Spec.~Radiat.~Transf.}}
                % Journal of Quantitiative Spectroscopy and Radiative Transfer
\def\memsai{\ref@jnl{Mem.~Soc.~Astron.~Italiana}}
                % Mem. Societa Astronomica Italiana
\def\nphysa{\ref@jnl{Nucl.~Phys.~A}}   % Nuclear Physics A
\def\physrep{\ref@jnl{Phys.~Rep.}}   % Physics Reports
\def\physscr{\ref@jnl{Phys.~Scr}}   % Physica Scripta
\def\planss{\ref@jnl{Planet.~Space~Sci.}}   % Planetary Space Science
\def\procspie{\ref@jnl{Proc.~SPIE}}   % Proceedings of the SPIE

\let\astap=\aap
\let\apjlett=\apjl
\let\apjsupp=\apjs
\let\applopt=\ao


\makeatother

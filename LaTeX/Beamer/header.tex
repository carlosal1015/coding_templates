\documentclass[11pt]{beamer}





\usepackage{lmodern} 		% Diese beiden packages sorgen für echte 
\usepackage[T1]{fontenc}	% Umlaute.

\usepackage{amssymb, amsmath, color, graphicx, float, setspace, tipa}
\usepackage[utf8]{inputenc} 
\usepackage[english]{babel}
\usepackage[justification=centering]{caption}
\addto\captionsenglish{\renewcommand{\figurename}{}} %Abbildungen nicht bzw. anders beschriften.


%\usepackage[pdfpagelabels,pdfstartview = FitH,bookmarksopen = true,bookmarksnumbered = true,linkcolor = black,plainpages = false,hypertexnames = false,citecolor = black, breaklinks]{hyperref}
%\usepackage{url}
\usepackage{picins} 		%Gleittext um Grafik. Befehl: parpic. Vorlage siehe unten
\usepackage{longtable} 		%Seitenübergreifende Tabelle. Vorlage siehe unten
\newtheorem*{bem}{Bemerkung} % Neue Theorem-Umgebung: Bemerkung
\newcommand{\fillframe}{\vskip0pt plus 1filll} 
\newcommand{\musr}{$\mu$SR }


%-----------------
%BEAMER-SPEZIFISCH
%-----------------

\usefonttheme{professionalfonts} % using non standard fonts for beamer
\usefonttheme{serif} % default family is serif


\usetheme{default}
% Verschiedene Varianten von usetheme, usecolortheme und usefonttheme kann man hier ausprobieren: http://deic.uab.es/~iblanes/beamer_gallery/

% \usetheme{
% 	AnnArbor | Antibes | Bergen |
% 	Berkeley | Berlin | Boadilla |
% 	boxes | CambridgeUS | Copenhagen |
% 	Darmstadt | default | Dresden |
% 	Frankfurt | Goettingen |Hannover |
% 	Ilmenau | JuanLesPins | Luebeck |
% 	Madrid | Malmoe | Marburg |
% 	Montpellier | PaloAlto | Pittsburgh |
% 	Rochester | Singapore | Szeged |
% 	Warsaw
% }
%Interessant scheinen: Boadilla, boxes, CambridgeUS, default, (Goettingen), Hannover, Madrid, Montpellier, Pittsburgh, Rochester, Singapore, Szeged, 

\usecolortheme{dove}
% \usecolortheme{
% 	albatross | beaver | beetle |
% 	crane | default | dolphin |
% 	dove | fly | lily | orchid |
% 	rose |seagull | seahorse |
% 	sidebartab | structure |
% 	whale | wolverine
% }

\usefonttheme{structurebold}
% 	default | professionalfonts | serif |
% 	structurebold | structureitalicserif |
% 	structuresmallcapsserif
% }


%\useinnertheme{
% 	circles | default | inmargin |
% 	rectangles | rounded
% } Am besten sein lassen.


% \useoutertheme{
% 	default | infolines | miniframes |
% 	shadow | sidebar | smoothbars |
% 	smoothtree | split | tree
% } Am besten sein lassen.



\setbeamercovered{transparent} %Halbtransparente Overlays (was als nächstes Element auf der Folie gezeigt wird)
\beamertemplatenavigationsymbolsempty % Entfernt Navigationssymbole unten
%\setbeamertemplate{footline}[frame]  % Seitenzahlen
    \setbeamertemplate{footline}{%
    	\raisebox{5pt}{\makebox[\paperwidth]{\hfill\makebox[10pt]{\hyperlink{tableofcontents}{\scriptsize\insertframenumber}}}}}


\documentclass[10pt]{article}

%% Text-Encoding festlegen. Mit utf8 oder utf8x funktionieren Umlaute wie gewohnt.
%% (mit Bibtex funktioniert nur utf8)
\usepackage[utf8x]{inputenc}

%% Sprachdatei für Trennregeln, Datum-Format und ähnliches festlegen
\usepackage[german]{babel}  % nötig für Umlaute
% \usepackage[english]{babel}

%% optimiert das typographische Erscheinungsbild
\usepackage{microtype}

%% erlaubt Listen einfacher zu formatieren (bietet nosep für kompakte Listen)
\usepackage{enumitem}
%% erlaubt hübsche Tabellen über mehrere Seiten, beinhaltet booktabs (\toprule, \midrule, ...)
\usepackage{ctable}
%% ermöglicht farbigen Text ({\color{red} ...})
\usepackage{xcolor}

%% erweiterte Funktionalität für Formeln (Pakete der American Mathematical Society)
\usepackage{amsfonts,amsmath,amsthm,amssymb}

%% vordefinierte Einheiten, einfaches Angeben von Einheiten (\SI{8 \pm 1}{cm})
%%   die Unsicherheit soll mit +- abgetrennt werden
\usepackage[separate-uncertainty]{siunitx}
\sisetup{
    range-units = single,       % \SIrange soll die Einheit nur einmal anzeigen
    list-units  = repeat,       % \SIlist soll die Einheit wiederholen
}
%% bei siuntix funktioniert babel leider nicht
%% für englische Dokumente sollten diese Zeilen auskommentiert werden.
\sisetup{
    range-phrase         = { bis },
    list-final-separator = { und },
%    list-pair-separator  = { und }, % an Uni noch nicht verfügbar
}

%% erlaubt es Bilddateien einzubinden
%% (ctable graphicx intern auch. Trotzdem ist es sinnvoll graphicx expilizt zu laden.
%%  Sonst entstehen schwehr verständliche Fehler, wenn ctable entfernt wird)
\usepackage{graphicx}
%% ermöglicht Bilder und Tabellen am eingegebenen Ort zu platzieren ([H])
\usepackage{float}
%% ermöglicht Unter-Bilder in einer figure-Umgebung
\usepackage{subfig}
%% Grafik-Dateien werden in den folgenden Ordnern gesucht
\graphicspath{{img/}}
%% Grafikdateien haben die folgenden Endungen (höchste Priorität zu erst)
\DeclareGraphicsExtensions{.pdf,.png,.jpg}

%% Vertikaler Abstand zwischen Absätzen, Beginn eines Absatzes nicht einrücken
\usepackage{parskip}
% \setlength{\parskip}{0.6em}   % Vertikaler Abstand zwischen Absätzen anpassen
% \setlength{\parindent}{0em}   % Einrück-Abstand anpassen

%% zeige Labels im Seitenrand. Dies ist praktisch um Verweise zu kontrollieren
\usepackage[final]{showkeys} % die Option 'final' deaktiviert die Ausgabe von showkeys

%% Seiten-Layout einstellen
\usepackage[
 a4paper,
 total={16cm,26cm},          % Breite und Höhe des Inhalt-Bereichs
 top=20mm, left=30mm,        % Ränder oben und links
 headsep=10mm,               % Abstand des unteren Rands der Kopfzeile vom oberen Rand des Inhalts
 footskip=10mm               % Abstand des unteren des Inhalts zum oberen Rand der Fusszeile
]{geometry}

%% Ermöglicht Links im PDF
%%   sollte möglichst spät in der Präambel geladen werden
\usepackage[
 pdftex,                        % wir verwenden pdftex/pdflatex
 bookmarks=true,                % wir wollen auch im PDF-Reader ein Inhaltsverzeichnis
 bookmarksdepth=3,              % das Inhaltsverzeichnis soll 3 Tiefen enthalten
 colorlinks=true,               % Linktexte sollen Farbig sein
 linkcolor=black,               % Links innerhalb des Dokuments bleiben schwarz
 citecolor=black,               % Links zu Quellenangaben bleiben ebenfalls schwarz
 urlcolor=blue,                 % URL-Linktexte sollen blau dargestellt werden
%  pdfborder={0 0 0}              % Links im PDF erhalten keinen Rahmen, nur nötig wenn colorlinks=false
]{hyperref}

%% definiert \cref: Referenzen mit korrekter Bezeichnung (z.B. "Abbildung 1")
%%   die Nummer alleine ist weiter mittels \ref verfügbar
%% muss NACH 'hyperref' geladen werden
\usepackage[german]{cleveref}
% \usepackage[english, capitalise]{cleveref}


%% Angaben für \maketitle
\title{Die \LaTeX-Beispieldatei}
\author{Nicola Chiapolini}
% \date{7. Mai 2013}             % ohne Angabe wird das heutig Datum verwendet

%% Angaben für die PDF-Eigenschaften
\makeatletter
\hypersetup{
  pdfauthor = {\@author},
  pdftitle = {\@title},
}
\makeatother

\begin{document}

\maketitle

\begin{abstract}
Diese Beispieldatei enthält eine Präambel mit den wichtigsten Befehle
  für einen normalen Artikel.
Ausserdem illustriert sie die verschiedenen Pakete jeweils mit
  einem entsprechenden Inhaltselement.
\end{abstract}

\tableofcontents

\section{Ein paar Tipps \& Tricks}
Ein nützlicher Tipp fürs schreiben von LaTeX Quelltext, ist es jeden
  Satz auf einer neuen Zeile zu beginnen.
Bei langen Zeilen kann der Satz umgebrochen und eingerückt werden.
Dies macht es einfach einen Satz aus dem PDF auch im Quelltext
  wiederzufinden.
Denn meist wirst du das PDF fürs Korrekturlesen verwenden.

\subsection{Die wichtigsten Inhaltselemente}
Hier folgen nun verschiedene Inhaltselemente.
Als erstes zeigt \cref{eq:beispiel} den Satz des Pythagoras.
Wie man Formeln setzt, lernst du wohl am Besten mit Kiles
  umfangreicher Symboltabelle.
\begin{equation}
\label{eq:beispiel} a^2 + b^2 = c^2
\end{equation} 

Im zweiten Absatz dieses Abschnitts folgt nun eine Aufzählung.
Sie erklärt dir, was du machen musst, damit \cref{fig:beispiel}
kompiliert:
\begin{enumerate}[nosep]
\item Erstelle einen Unterordner \texttt{\color{red} img} (vgl. \texttt{graphicspath})
\item Lade das Beispielbild\\
  \url{http://www.physik.uzh.ch/~nchiapol/info1/res/iss+spaceshuttle.jpg}\\
  herunter und speichere es in diesen Ordner.
\end{enumerate}

\begin{figure}[H]
 \centering
 %% Die dreifachen Klammern {{{ }}} sind nötig, damit Latex auch mit 
 %% Dateinamen umgehen kann, die Punkte enthalten.
 \includegraphics[width=0.45\textwidth]{{{iss+spaceshuttle}}}
 \caption{Das Spaceshuttel und die ISS}
 \label{fig:beispiel}
\end{figure}

Die Grösse des Bildes ist in unserem Fall in Bruchteilen der 
  Textbreite angegeben. 
Das ist oft die beste Art.
Sie könnte aber zum Beispiel mit \texttt{[width=8cm]} auch als
  \SI{8}{\centi\meter} angegeben werden.

Mit den \SI{8}{cm} haben wir auch gleich gezeigt wie man mit \texttt{siunitx} 
  verwendet. 
Das Paket versteht auch Zahlen mit Fehlern wie \SI{60 \pm 5}{kg}, 
Wertebereiche wie \SIrange{5}{10}{\degreeCelsius} oder Listen \SIlist{1;2;3;4}{s}.

Schliesslich sollten wir auch noch eine einfache Tabelle 
  zu unserem Beispiel hinzufügen.
Wir erlauben \cref{tab:beispiel} aber selbst ihre Position zu wählen.
\begin{table}[tph]
\centering
\begin{tabular}{lc}
\toprule
Name & Wert\\
\midrule
eins & 1\\
zwei & 2\\
drei & 3\\
vier & 4\\
\bottomrule
\end{tabular}
\caption{Unsere Beispiel-Tabelle}
\label{tab:beispiel}
\end{table} 

\subsection*{Quellenangaben}
Der \texttt{*} macht dies zu einem unnummerierten Absatz.
Nun aber zu den versprochenen Quellenangaben. 
Um zu zeigen, wie man in Latex Quellen zitieren kann, 
  sei hier auf ein Hilfsmittel im Web verwiesen.
Immer wieder nützlich ist das Wiki-Book zu LaTex \cite{wikibook}.
Es ist sehr umfangreich und enthält viele nützliche Tipps.

\begin{thebibliography}{100}
\addcontentsline{toc}{section}{Bibliography}
\bibitem{wikibook} LaTeX, \url{http://en.wikibooks.org/wiki/LaTeX}
\end{thebibliography}

\end{document}
